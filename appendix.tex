% !TEX root = main.tex
%////////////////////////////////////////////////////////
\begin{center}
\section*{\kintou{2.5zw}{付録}}                      %% ここに番号をつけない
\vspace*{-2zh}
\end{center}
\addcontentsline{toc}{section}{付録} %% 目次に番号をつけない
\appendix

\setcounter{equation}{0}
\setcounter{figure}{0}
\setcounter{table}{0}

\makeatletter
     \renewcommand{\theequation}{%
          A.\arabic{equation}}
     \@addtoreset{equation}{section}
\makeatother

\makeatletter
     \renewcommand{\thetable}{%
          A.\arabic{table}}
     \@addtoreset{table}{section}
\makeatother

\makeatletter
     \renewcommand{\thefigure}{%
          A.\arabic{figure}}
     \@addtoreset{figure}{section}
\makeatother

% ================================================
\makeatletter
     \renewcommand{\thesubsection}{%
          A.\arabic{subsection}}
     \@addtoreset{subsection}{section}
\makeatother

\makeatletter
     \renewcommand{\thesubsubsection}{%
          A.\arabic{subsection}.\arabic{subsubsection}}
     \@addtoreset{subsubsection}{section}
\makeatother

% ================================================
\makeatletter
     \renewcommand{\thetheorem}{%
          A.\arabic{theorem}}
     \@addtoreset{theorem}{section}
\makeatother

\makeatletter
     \renewcommand{\thedefinition}{%
          A.\arabic{definition}}
     \@addtoreset{definition}{section}
\makeatother

\makeatletter
     \renewcommand{\thelemma}{%
          A.\arabic{lemma}}
     \@addtoreset{lemma}{section}
\makeatother

\makeatletter
     \renewcommand{\thelstlisting}{%
          {\bf{A.\arabic{lstlisting}}}}
     \@addtoreset{lstlisting}{section}
\makeatother






%////////////////////////////////////////////////////////

%%%%%%%%%%%%%%%%%%%%%%%%%%%%%%%%%%%%%%%%%%%%%%%%%%%%%%
\subsection{ちちち}
%%%%%%%%%%%%%%%%%%%%%%%%%%%%%%%%%%%%%%%%%%%%%%%%%%%%%%
 ここに,本文を書く.ここに,本文を書く.ここに,本文を書く.ここに,本文を書く.
ここに,本文を書く.ここに,本文を書く.ここに,本文を書く.ここに,本文を書く.
ここに,本文を書く.ここに,本文を書く.ここに,本文を書く.ここに,本文を書く.
ここに,本文を書く.ここに,本文を書く.ここに,本文を書く.ここに,本文を書く.
ここに,本文を書く.ここに,本文を書く.ここに,本文を書く.ここに,本文を書く.
ここに,本文を書く.ここに,本文を書く.ここに,本文を書く.ここに,本文を書く.
ここに,本文を書く.ここに,本文を書く.ここに,本文を書く.ここに,本文を書く.
ここに,本文を書く.ここに,本文を書く.ここに,本文を書く.ここに,本文を書く.
ここに,本文を書く.ここに,本文を書く.ここに,本文を書く.ここに,本文を書く.
ここに,本文を書く.ここに,本文を書く.ここに,本文を書く.ここに,本文を書く.
ここに,本文を書く.ここに,本文を書く.ここに,本文を書く.ここに,本文を書く.
ここに,本文を書く.ここに,本文を書く.ここに,本文を書く.ここに,本文を書く.
ここに,本文を書く.ここに,本文を書く.ここに,本文を書く.ここに,本文を書く.
ここに,本文を書く.ここに,本文を書く.ここに,本文を書く.ここに,本文を書く.


%%%%%%%%%%%%%%%%%%%%%%%%%%%%%%%%%%%%%%%%%%%%%%%%%%%%%%
\subsection{つつつつつつつつつつつつつつつつつつつつつつつつつつつつつつつつつつ}
%%%%%%%%%%%%%%%%%%%%%%%%%%%%%%%%%%%%%%%%%%%%%%%%%%%%%%
\subsubsection{ててて}
%%%%%%%%%%%%%%%%%%%%%%%%%%%%%%%%%%%%%%%%%%%%%%%%%%%%%%
 ここに,本文を書く.ここに,本文を書く.ここに,本文を書く.ここに,本文を書く.
ここに,本文を書く.ここに,本文を書く.ここに,本文を書く.ここに,本文を書く.
ここに,本文を書く.ここに,本文を書く.ここに,本文を書く.ここに,本文を書く.
ここに,本文を書く.ここに,本文を書く.ここに,本文を書く.ここに,本文を書く.
ここに,本文を書く.ここに,本文を書く.ここに,本文を書く.ここに,本文を書く.
ここに,本文を書く.ここに,本文を書く.ここに,本文を書く.ここに,本文を書く.
ここに,本文を書く.ここに,本文を書く.ここに,本文を書く.ここに,本文を書く.
ここに,本文を書く.ここに,本文を書く.ここに,本文を書く.ここに,本文を書く.
ここに,本文を書く.ここに,本文を書く.ここに,本文を書く.ここに,本文を書く.
ここに,本文を書く.ここに,本文を書く.ここに,本文を書く.ここに,本文を書く.
ここに,本文を書く.ここに,本文を書く.ここに,本文を書く.ここに,本文を書く.
ここに,本文を書く.ここに,本文を書く.ここに,本文を書く.ここに,本文を書く.
ここに,本文を書く.ここに,本文を書く.ここに,本文を書く.ここに,本文を書く.
ここに,本文を書く.ここに,本文を書く.ここに,本文を書く.ここに,本文を書く.
ここに,本文を書く.ここに,本文を書く.ここに,本文を書く.ここに,本文を書く.
ここに,本文を書く.ここに,本文を書く.ここに,本文を書く.ここに,本文を書く.

%%%%%%%%%%%%%%%%%%%%%%%%%%%%%%%%%%%%%%%%%%%%%%%%%%%%%%
\subsubsection{ととととととととととととととととととととととととととととととととととととととととととと}
%%%%%%%%%%%%%%%%%%%%%%%%%%%%%%%%%%%%%%%%%%%%%%%%%%%%%%
 ここに,本文を書く.ここに,本文を書く.ここに,本文を書く.ここに,本文を書く.
ここに,本文を書く.ここに,本文を書く.ここに,本文を書く.ここに,本文を書く.
ここに,本文を書く.ここに,本文を書く.ここに,本文を書く.ここに,本文を書く.
ここに,本文を書く.ここに,本文を書く.ここに,本文を書く.ここに,本文を書く.
ここに,本文を書く.ここに,本文を書く.ここに,本文を書く.ここに,本文を書く.
ここに,本文を書く.ここに,本文を書く.ここに,本文を書く.ここに,本文を書く.
ここに,本文を書く.ここに,本文を書く.ここに,本文を書く.ここに,本文を書く.
ここに,本文を書く.ここに,本文を書く.ここに,本文を書く.ここに,本文を書く.
ここに,本文を書く.ここに,本文を書く.ここに,本文を書く.ここに,本文を書く.
ここに,本文を書く.ここに,本文を書く.ここに,本文を書く.ここに,本文を書く.
ここに,本文を書く.ここに,本文を書く.ここに,本文を書く.ここに,本文を書く.
ここに,本文を書く.ここに,本文を書く.ここに,本文を書く.ここに,本文を書く.
ここに,本文を書く.ここに,本文を書く.ここに,本文を書く.ここに,本文を書く.
ここに,本文を書く.ここに,本文を書く.

%jlistingが文字化けしてるので省略
% ==============================================
% \lstinputlisting[%
% 	caption=\hspace{1zw}{{\tt sample.c}:正負の判定},%
% 	label={code:sample}]%
% 	{program/sample.c}%
% ==============================================





