% !TEX root = main.tex
\section{緒言}
近年,日本では少子高齢化の進展に伴い,生産年齢人口の減少が深刻な社会問題となっている.
特に農業分野では,熟練した高齢者が農作業の多くを担っている現状があり,
後継者不足が課題として浮き彫りになっている.
農林水産省の統計によれば,日本の農業従事者の平均年齢は約67歳に達しており,
農業人口の減少とともに農業現場の担い手不足が顕著化している.
このような背景から,効率的な農作業を実現するために,
自動化技術やロボット技術への期待が高まっている\cite{nougyou}.

農業ロボットには,耕運,草刈り,収穫,搬送といった
多岐にわたるタスクを自律的に遂行する能力が求められており,
これにより農作業の省力化や効率化が期待されている.
特に収穫作業においては,人が収穫した作物を運搬する搬送ロボットの需要が高い\cite{nougyouziritu}.
このような搬送ロボットには,収穫者を正確に追従しながら作業をサポートする能力が必要である.

現在市販されている多くの搬送ロボットには,LiDARや3Dセンサーなどの高価なセンサーが搭載されている場合が多い.
LiDARは高精度な距離計測を可能にし,障害物回避や人追従において重要な役割を果たす一方で,
その導入コストが非常に高く,特に中小規模の農業現場では採用が難しい.
この課題を克服するため,
近年ではカメラ映像を用いて低コストかつ高精度な追従を実現する技術の研究が進められている.
例えば,カメラ映像のみで物体位置を認識し,
自律的に移動するロボット車両の開発が行われている\cite{saga-u}.

本研究では,農作物収穫時に収穫者の後を追従し,農作物を入れる籠を搬送するロボットを念頭に研究を行う.
具体的にはカメラを用いて作業者を認識し,作業者の移動に追従させるアルゴリズムを構築する.
人追従には飛翔物の誘導制御をする際に用いられる比例航法\cite{hirei}\cite{roshon}を基礎として改良を加えた手法を提案する.
また,提案手法を実環境下で評価し,その追従性能および動作の滑らかさを検証する.
以上のことにより,低コストで高精度な追従を実現する移動ロボットの可能性を示す.