% !TEX root = main.tex
\section{はじめに}

近年,生産年齢人口の減少が社会的な課題となっており,
ロボット技術の導入による作業の効率化が期待されている.
例えば,草刈りロボットや運搬ロボットなど,人手不足を補うための自律ロボットが徐々に普及しつつある.

しかしながら,現在出回っている多くの搬送ロボットには,
LiDAR(Light Detection and Ranging)や3Dセンサーなどのセンサーが搭載されているケースが多い.
LiDARは非常に精度の高い距離計測を可能にし,障害物回避や人追従において重要な役割を果たすが,
その一方で,導入費用が高いため,特に中小企業や農業の現場では導入が難しい現状がある.
このため,より低コストで同様の機能を実現する技術が求められている.

この問題に対処するため,本研究ではカメラのみを用いた人追従型ロボットの開発を目指す.
カメラを使用することで,より安価でかつ軽量なシステムの構築が可能であり,初期コストの削減が期待される.
また,カメラ技術の進展により,物体認識や追従の精度も向上しており,
特定の対象を認識し追従するアルゴリズムも開発されている\cite{roshon}.
しかしながら実機で検証が行われた例は少ない.

本研究では,カメラを用いた人追従アルゴリズムを構築後,実際の環境下で実験を行い,
追従性能の精度を検証する.
