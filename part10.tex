% !TEX root = main.tex
\section{緒言}

近年,生産年齢人口の減少が社会的な課題となっており,
ロボット技術の導入による作業の効率化が期待されている.
例えば,草刈りロボットや運搬ロボットなど,人手不足を補うための自律ロボットが徐々に普及しつつある\cite{saga-u}.

しかしながら,現在出回っている多くの搬送ロボットには,
多くの自動搬送ロボットにはLiDARや3Dセンサーなどの高価なセンサーが搭載されているケースが多い.
LiDARは高精度な距離計測を可能にし,障害物回避や人追従において重要な役割を果たす一方で,その導入コストが非常に高く,
特に中小企業や農業現場では導入が難しい現状がある.
近年,カメラ映像を用いて物体位置を認識し,自律移動を行うシステムは物流や農業,介護など幅広い分野で需要が高まっている\cite{saga-u}\cite{roshon}.

本研究では,カメラを用いた人追従アルゴリズムを構築し,比例航法\cite{hirei}を基礎として改良を加えた手法を提案する.
さらに,提案手法を実環境下で評価し,その追従性能および動作の滑らかさを検証する.
これにより,低コストで高精度な追従を実現する移動ロボットの可能性を示すことを目的とする.