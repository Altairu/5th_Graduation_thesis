% !TEX root = main.tex
\section{結語}
本研究では,カメラのみを用いた人追従搬送ロボットの開発を目指し,
Intel RealSense D435iから取得したデータを活用して人の位置を推定し,
追従アルゴリズムに基づいてロボットを制御するシステムを構築した.
特に,修正比例航法(MPN)やゲインスケジューリング下修正比例航法(GS-MPN)を用いることで,
従来の比例航法(PN)では対応が難しかった滑らかな動作や安定性の向上を図った.
また,デプスカメラの画像データを用いて物理空間での人の座標を高精度に推定する仕組みを実装し,
ロボット制御に反映させた.これにより,低コストなセンサーを使用したロボットシステムの可能性を示すことができた.

本研究の成果として,次の点が挙げられる.まず,PN,MPN,GS-MPNの3つの追従アルゴリズムを比較評価し,
GS-MPNが最もバランスの取れた追従性能と動作の滑らかさを実現することを確認した.
また,ロボットのハードウェア構成として,D435iを中心としたカメラシステム,STM32を使用した下位制御層,
およびROS2による上位制御層を統合したシステムを構築し,それぞれの役割を分担させることで,
スムーズなデータ処理と制御を可能にした.さらに,追従性能の評価では,
GS-MPNが急加速やズレの少ない安定した追従を実現し,従来手法を上回る性能を示した.

一方で,本研究にはいくつかの課題も残された.まず,カメラの視野内での追従には成功したものの,
遮蔽物が発生した場合や複数の対象物が存在する場合における動作安定性については十分に検証が行えていない.
特定のターゲットを継続的に追従するための識別アルゴリズムが必要である.
さらに,外乱環境下での耐性についても課題が残る.例えば,屋外での風や路面の不整に対する追従性能の検証は未実施であり,
システムの実用化に向けてはこれらの外乱要因への対応が求められる.

また,本研究では比例航法を用いて追従制御を行っが,歩行経路の予測を導入することで,
人の進行方向に先回りする制御を実現し,応答性を向上させることも今後の課題として挙げられる.
