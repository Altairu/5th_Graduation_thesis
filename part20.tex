% !TEX root = main.tex

\section{搬送ロボット制御}
本章では搬送ロボットの制御について説明する.
本研究では,その制御が簡単にできるということから搬送ロボットとして2輪ロボットを用いることとした.
以下では2輪ロボットの運動学とその制御手法について説明する.

\subsection{2輪ロボットの運動学}
2輪ロボットは,シンプルな構造でありながら,高い機動性を持つため,広く研究や産業に利用されている.
その運動学は,ロボットの速度制御や経路計画を行う上で不可欠な要素である.
本節では,2輪ロボットの運動学および逆運動学を導出する.
順運動学を導出する際に用いるモデルを図\ref{fig:robot_model}に示す.
また,図中の変数は以下の通りである.

\begin{itemize}
    \item $v$:ロボットの並進速度 [m/s]
    \item $\omega$:ロボットの角速度 [rad/s]
    \item $r$:車輪の半径 [m]
    \item $L$:左右車輪間の距離 [m]
    \item $\dot{\theta}_R$,$\dot{\theta}_L$:右車輪および左車輪の角速度 [rad/s]
\end{itemize}

\begin{figure}[h]
    \centering
    \includegraphics[width=0.5\textwidth]{figure/robot_model.pdf}
    \caption{2輪ロボットのモデル図}
    \label{fig:robot_model}
\end{figure}


\subsubsection{順運動学}
順運動学では,各車輪の角速度$\dot{\theta}_R$および$\dot{\theta}_L$から,ロボットの並進速度$v$,角速度$\omega$,および自己位置を算出する.
自己位置の計算は,ロボットの進行方向と回転を考慮して行う.

ロボットの並進速度$v$と角速度$\omega$は,次式で求められる.
\begin{equation}
    v = \frac{r}{2} (\dot{\theta}_R + \dot{\theta}_L)
    \label{eq:forward_v}
\end{equation}
\begin{equation}
    \omega = \frac{r}{L} (\dot{\theta}_R - \dot{\theta}_L)
    \label{eq:forward_omega}
\end{equation}

したがって,ロボットの$(x, y)$方向の速度と角速度は次式で得られる.
\begin{equation}
    \dot{x} = v \cos \theta
    \label{eq:forward_x}
\end{equation}
\begin{equation}
    \dot{y} = v \sin \theta
    \label{eq:forward_y}
\end{equation}
\begin{equation}
    \dot{\theta} = \omega
    \label{eq:forward_theta}
\end{equation}
ただし,本研究ではマイコン等により制御を行うため,これらの微分方程式を離散化することで,時刻$t_k$における位置と向きを計算する.
なお,離散化した式は次式となる.
\begin{equation}
    x_{k+1} = x_k + v_k \cos \theta_k \Delta t
    \label{eq:discrete_x}
\end{equation}
\begin{equation}
    y_{k+1} = y_k + v_k \sin \theta_k \Delta t
    \label{eq:discrete_y}
\end{equation}
\begin{equation}
    \theta_{k+1} = \theta_k + \omega_k \Delta t
    \label{eq:discrete_theta}
\end{equation}
ここで,$\Delta t$は計算ステップの時間間隔を示す.


\subsubsection{逆運動学}
逆運動学では,ロボットの並進速度$v$と角速度$\omega$から,
車輪の角速度$\dot{\theta}_R$および$\dot{\theta}_L$を求める.
ロボットを動作させる際には目標値となる$\dot{\theta}_R$および$\dot{\theta}_L$が必要となるため,逆運動学を用いる必要がある.

順運動学により求めた式 \eqref{eq:forward_v}および式\eqref{eq:forward_omega}により,次式を得ることができる.
\begin{equation}
    \dot{\theta}_R = \frac{1}{r} (v + \frac{L}{2} \omega)
    \label{eq:theta_R}
\end{equation}
\begin{equation}
    \dot{\theta}_L = \frac{1}{r} (v - \frac{L}{2} \omega)
    \label{eq:theta_L}
\end{equation}

これらの式により,目標速度$v$と$\omega$が与えられた場合に,
車輪の角速度$\dot{\theta}_R$と$\dot{\theta}_L$を算出し,ロボットの各車輪を制御することができる.

\subsection{PID制御とローパスフィルタ}
PID制御(Proportional-Integral-Derivative Control)は制御工学における最も基本的かつ広く使用されているフィードバック制御の手法である.
本研究でも車輪の角速度制御にPID制御を用いることとした.
本節では,PID制御の基本構造とノイズ除去に使用するローパスフィルタについて説明する.


\subsubsection{PID制御の構造}
PID制御は,目標値と現在値の偏差を基に制御量を算出し,システムを目標値へ収束させる手法である.
基本的なPID制御手法において制御量$u(t)$は次式で定義される:
\begin{equation}
    u(t) = K_P e(t) + K_I \int_{0}^{t} e(\tau) d\tau + K_D \frac{de(t)}{dt}
    \label{eq:pid}
\end{equation}
ここで,式中の変数は以下の通りである.
\begin{itemize}
    \item $e(t)$:目標値と現在値の偏差
    \item $K_P$:比例ゲイン(P制御)
    \item $K_I$:積分ゲイン(I制御)
    \item $K_D$:微分ゲイン(D制御)
\end{itemize}
また,式中の各項の役割は以下の通りである.
\begin{itemize}
    \item 比例制御(P制御):偏差に比例した制御量を生成し,応答の速さを向上させる.
    \item 積分制御(I制御):偏差を累積することで,定常偏差を排除する.
    \item 微分制御(D制御):偏差の変化率を基に,過渡応答の振動を抑制する.
\end{itemize}

\begin{figure}[h]
    \centering
    \includegraphics[width=0.7\textwidth]{figure/pid.pdf}
    \caption{PID制御器の構成図}
    \label{fig:pid_controller}
\end{figure}

図\ref{fig:pid_controller}に本研究で使用するPID制御器の構成図を示す.
微分項においてはノイズの影響により過大な入力を生成する可能性があるため,
本研究では不完全微分を用いることとした.これは実質的にローパスフィルタを導入することと同義である.
次節ではローパスフィルタについて説明する.


\subsubsection{ローパスフィルタの導入}
前節でも述べた通り,微分制御(D制御)は高周波ノイズに敏感であり,そのまま使用するとシステムの安定性を損なう可能性がある.
この問題を軽減するために,微分項にローパスフィルタを適用することが一般的である.

ここで,ローパスフィルタを適用した微分制御項は次式で表される.
\begin{equation}
    D(s) = \frac{K_D s}{1 + T s}
    \label{eq:lowpass_filter}
\end{equation}
ただし,$T$はカットオフ周波数である.
カットオフ周波数を適切に設定することで,ローパスフィルタにより高周波ノイズを低減することができ,
制御系への影響を軽減することができる.

\section{物体認識と座標検出}
本研究ではカメラによる人追従制御を行うが,当然ながらカメラ画像から人の認識を行い,
ロボットが進むべき座標を導出する必要がある.
本章では本研究で使用した物体認識手法と座標検出について説明する.

\subsection{YOLOv5による物体認識}
本研究では,人の認識にYOLOv5(You Only Look Once Version 5)を使用する.
YOLOはリアルタイムで物体を検出するアルゴリズムであり,
1枚の画像に対して物体のクラスとその位置を同時に出力する特徴を持つ.
単一のニューラルネットワークを用いて,画像内の物体のバウンディングボックスとクラスラベルを直接予測する\cite{yolo}.
そのため他の物体検出モデル(例えばSSDやFaster R-CNN)と比較して,
高速な推論速度と高い精度を両立している.このため,リアルタイム性が要求されるロボットシステムにおいて有用である.

ここでは,YOLOv5を用いてデプスカメラD435iから得られたRGB画像内の人を認識することを考える.
YOLOv5は以下のステップで物体を検出する.
\begin{enumerate}
    \item RGB画像を入力し,YOLOv5のバックボーンにより特徴を抽出する.
    \item ネックでスケールを調整し,特徴を統合する.
    \item ヘッドでバウンディングボックス,クラスラベル(人),および信頼度スコアを出力する.
\end{enumerate}

出力されたバウンディングボックスは,ピクセル座標で記述されており,
これをデプスカメラのデプスマップと組み合わせることで,物理座標を算出することができる.

\subsection{デプスデータからの人座標検出}
デプスカメラは,対象物までの距離を計測するためのデバイスであり,ロボットの環境認識や障害物回避,
物体追跡などの応用において重要な役割を果たす.
本研究では,インテル® RealSense™ デプスカメラ D435iを使用し,
カメラから取得したデプスデータを基に人追従アルゴリズムを実現する.

デプスカメラで取得したデータを用いて,画像上のピクセル座標を実空間の物理座標に変換することで,人の位置を検出する.
ピクセル座標系における対象物の位置をカメラ座標系(図\ref{fig:coordinate_conversion})に変換する手順を以下に述べる.

\begin{figure}[h]
    \centering
    \includegraphics[width=0.4\textwidth]{figure/rialsens_man.pdf}
    \caption{デプスカメラを用いた座標変換モデル}
    \label{fig:coordinate_conversion}
\end{figure}

デプスカメラのデプスマップは,各ピクセルに深度情報$L$(対象物までの斜距離)を格納している.
この情報を用いて,画像座標系におけるピクセル座標$(u, v)$をカメラ座標系$(X_c, Y_c)$に変換する.
なお,本研究では人に追従すれば良いため,上下方向($Z$軸)はを考慮する必要はない.


以下の式を使用することで座標変換することができる.
\begin{equation}
    X_c = \frac{(u - c_x) \cdot L}{f_x}
    \label{eq:xc}
\end{equation}
\begin{equation}
    Y_c \approx L
    \label{eq:yc}
\end{equation}
ここで,$u$および$v$は画像座標系のピクセル位置,$c_x$はカメラの光学中心
(水平方向の画面中心)のピクセル位置,
$f_x$はカメラの焦点距離(ピクセル単位),
$L$はデプスマップから取得される深度値(対象物までの斜距離)である.

つぎに,カメラ座標系における2次元位置$(X_c, Y_c)$をロボット座標系$(X_r, Y_r)$に変換する.
この変換は,カメラとロボット間の座標系の相対的な位置関係を記述する回転行列$R$と平行移動ベクトル$T$を用いることで行われる.

ロボット座標系への変換は,次の行列式で表される.
\begin{equation}
    \begin{bmatrix}
        X_r \\ Y_r \\ 1
    \end{bmatrix}
    =
    \begin{bmatrix}
        R & T \\ 
        0 & 1
    \end{bmatrix}
    \begin{bmatrix}
        X_c \\ Y_c \\ 1
    \end{bmatrix}
    \label{eq:transform}
\end{equation}
ここで,式中の変数は以下の通りである.
\begin{itemize}
    \item $R$:カメラ座標系からロボット座標系への回転行列
    \item $T$:カメラ座標系からロボット座標系への平行移動ベクトル
\end{itemize}

本研究では,カメラがロボットに平行に取り付けられているため,回転行列$R$は単位行列として扱う.
また,カメラがロボットの旋回中心から$200\ \mathrm{mm}$($0.2\ \mathrm{m}$)前方に取り付けられているため,
平行移動ベクトル$T$は以下のように与えられる.
\begin{equation}
    T = \begin{bmatrix} 0.0 \\ 0.2 \end{bmatrix}
    \label{eq:T}  % \labe -> \label に修正
\end{equation}

\eqref{eq:xc}~\eqref{eq:T}を用いた計算例を以下に示す.なお,式中パラメータを次のように仮定する.
\begin{itemize}
    \item デプスカメラの焦点距離:$f_x = 378.09$ [px]
    \item デプスカメラの光学中心:$c_x = 319.39$ [px]
    \item デプスマップからの深度値:$L = 2.5$ [m]
    \item 対象物の水平ピクセル座標:$u = 400$ [px]
    \item カメラ位置のオフセット:$T_x = 0.2$ [m]
\end{itemize}

まず,画像座標系でのズレ$\Delta u$を計算する.
\begin{equation}
    \Delta u = u - c_x = 400 - 319.39 = 80.61 \, \text{[px]}
\end{equation}

つぎに,カメラ座標系での水平座標$X_c$を求める.
\begin{equation}
    X_c = \frac{\Delta u \cdot L}{f_x} = \frac{80.61 \cdot 2.5}{378.09} \approx 0.534 \, \text{[m]}
\end{equation}

また,奥行き$Y_c$を近似的に求める.
\begin{equation}
    Y_c \approx L = 2.5 \, \text{[m]}
\end{equation}

最後に,ロボット座標系での位置$(X_r, Y_r)$を計算する.
\begin{equation}
    X_r = X_c + T_x = 0.534 +0 = 0.534 \, \text{[m]}
\end{equation}
\begin{equation}
    Y_r = Y_c + T_y = 2.5 + 0.2 = 4.5 \, \text{[m]}
\end{equation}

この計算により,目標物のロボット座標系における位置が得られる.

\newpage

\section{人追従アルゴリズム}
前章では人追従に必要となる座標を導出した.
本章では前章で求められた座標にロボットが移動するための制御手法を示す.
具体的には飛翔体の誘導に用いられる比例航法を搬送ロボットに適用することを考えた.

\subsection{比例航法 (Proportional Navigation, PN)}
比例航法は,目標物(人物)の追従において,距離と偏差(横方向のズレ)を用いて
制御信号を生成する基本的なアルゴリズムである.本来であれば,目標までの視線角度をもとに計算するが,簡易化のため,横方向のズレを使用する.
以下に,本実験で使用する比例航法の式を示す.

\begin{align}
    V      & = k_v \cdot (P_d - 1.0)                     \\
    \omega & = k_\omega \cdot \frac{P_o}{\max(P_d, 1.0)}
\end{align}
ただし,式中の変数は以下の通りである.
\begin{itemize}
    \item \(k_v\):並進速度の比例ゲイン
    \item \(k_\omega\):角速度の比例ゲイン
    \item \(P_d\):ロボットと目標間の距離
    \item \(P_o\):横方向の偏差
\end{itemize}

しかし,比例航法では対象までの距離が近づくにつれ横方向の偏差の変化が大きくなりロボットが振動的になると考えられる.
そこで次節で修正比例航法を提案する.

\subsection{修正比例航法 (Modified Proportional Navigation, MPN)}
比例方法では,大きな変化に対して大きな出力となり,ロボットの動きが激しくなってしまう.
そこで修正比例航法では,偏差の変化率(偏差角速度)を考慮することで,
ロボットの動作をより滑らかにする.
偏差角速度は以下の式で計算される.
\begin{equation}
    \dot{e}(k) = \frac{P_o(k) - P_o(k-1.0)}{\Delta t}
\end{equation}
また,角速度 \(\omega\) は以下の式で修正される.
\begin{equation}
    \omega(k) = k_\omega \cdot \frac{P_o(k)}{\max(P_d(k), 1.0)} + \lambda \cdot \dot{e}(k)
\end{equation}
ここで,式中の変数は以下の通りである.
\begin{itemize}
    \item \(\lambda\):偏差角速度に対するゲイン
    \item \(\Delta t\):制御周期
\end{itemize}

これにより,急激な方向転換や不安定な動作が軽減され,滑らかな動作が可能となると考えられる.
しかし,振動を抑えることで直線運動での速応性が低下することが懸念されるため,次節でさらなる手法についても考える.

\subsection{ゲインスケジューリング修正比例航法
    (Gain-Scheduled Modified Proportional Navigation, GS-MPN)}
前節の通り,直線運動での速応性が低下することが考えられるため,角速度に応じて$\lambda$を変化させることを考える.
そこで,ゲインスケジューリング修正比例航法を提案する.本手法では,偏差角速度に基づき動的なゲインスケジューリングを導入する.

前述の通り$\lambda$を動的微分ゲイン \(k_d\) として変更する. \(k_d\) を次式で定義する.
\begin{equation}
    k_d = \lambda_d \cdot \frac{1 - \exp(-a \cdot |\dot{\text{offset}}|)}{1 + \exp(-a \cdot |\dot{\text{offset}}|)}
\end{equation}
ただし,式中の変数は以下の通りである.
\begin{itemize}
    \item \(\lambda_d\):微分ゲインの最大値
    \item \(a\):微分ゲインの変化速度を調整するパラメータ
\end{itemize}
ここで,$a$を変化させた際の角速度に基づく動的微分ゲインのグラフを図\ref{fig:dynamicgain}に示す.

また,角速度 \(\omega\) は次式で計算される.
\begin{equation}
    \omega = k_\omega \cdot \frac{\text{person\_offset}}{\max(\text{person\_distance}, 1.0)} + k_d \cdot \dot{\text{offset}}
\end{equation}

\begin{figure}[H]
    \centering
    \includegraphics[width=0.7\textwidth]{figure/dynamicgain.pdf}
    \caption{動的微分ゲイン}
    \label{fig:dynamicgain}
\end{figure}

